\section*{Summary of the Proposal}

In recent years, the magnetoelastic coupling is emerging and growing up as a new research field. The coexisting state of magnons and phonons opens up tremendous possibilities, like modulating spin wave acoustically and tuning acoustic waves magnetically. This proposal is focusing on investigating the nonreciprocity of waves in magnon-phonon coupling state, targeting on the establishment of the new modulation method based on voltage controllable magnetic anisotropy (VCMA), and Dzyaloshinskii–Moriya interactions (DMI). By considering the importance of non-reciprocity in information transport and devices stabilization, this experiment is an essential building block for further understanding of the nonreciprocal response of waves and will lead to the multifunction future electronic devices.

\section*{Background}

Nonreciprocity is one of the most fundamental research themes for pushing the boundary of the known world. It discusses the unidirectional transport of the particles. Generally, it arises from nonlinearity \cite{TokuraNagaosaNR2018}, spatial inversion asymmetry and time-reversal asymmetry in noncentrosymmetric crystals \cite{DiNRSW2013,DiNRSW2015}, providing numerous tools for steering nature and advancing the technology. The most well-known example is the invention of the p-n diode, which triggered the Digital Revolution and undoubtedly reformed society. In digital technology, where waves carry the information, the unidirectional propagation is one of the most intensively investigated topics. Regarding signal transmission, nonreciprocity suppresses the backscattering of the wave, enhancing the transmission of information. Particularly, for the future magnonic memory devices, nonreciprocity of spin waves is the primary parameter for stabilization of circuits \cite{DiNRSW2015}.

Generally, the nonreciprocity of waves is examined accordingly depending on the wave types, such as acoustic waves, microwaves, spin waves, limiting the variety of the modulation methods. Towards future wave devices, interdisciplinary tools have undoubtedly promising potential for expanding the numbers of approaches. In fact, with the presence of magnetostrictive material, the mechanical oscillation, phonons, can couple with the magnons, driving magnetic dynamics, forming the magnetoelastic waves. Compared with the standard antenna excited spin waves, magnetoelastic waves can carry spin information for longer distance, due to the low damping. Also, based on magnetoelastic coupling, it has been proposed the acoustically assisted magnetic recording device~\cite{LiMS2014}. 

The acoustic wave devices are well-developed technique and commonly integrated into modern electronic devices, like sensors and microwave filters and wireless communication technologies, and also spin waves are under increasing interest for integrating with memory devices. With the sufficient well-established knowledge on acoustic waves, and spin waves and non-reciprocity, one can expect various new paths for the manipulation of non-reciprocity with the coexisting state of magnons with phonons, improving versatility of functions of the future wave-based devices.




\section*{Methods}

The magnetic control of acoustic wave nonreciprocity has been observed by our group Xu et al. and Sasaki et al.  \cite{SasakiNRMS2017,XuIEEMS2018}. It is highly influenced by the elastic tensor and magnetoelastic coupling coefficient tensor. At present, the origin of this magnetic field dependent nonreciprocal acoustic response is still ambiguous. In here, towards modulation and further understanding of nonreciprocity in the magnon-phonon coupling, we propose two modulation mechanisms, VCMA and DMI.             

Magnetoelastic coupling originates from the magnetoelastic anisotropy energy (ME). It is one of the main contributions to the total magnetic anisotropy energy, together with the magnetocrystalline(MC), the magnetostatic(MS), and the magnetic surface anisotropy energies. In general, these components are in a competitive relationship while MC dominates in most of the ferromagnetic materials. However, in the MgO/CoFeB/Ta stacking thin film, it has been reported the significant influence on magnetic perpendicular anisotropic from MS  \cite{GowthamPMA2016} while it also has giant voltage-induced perpendicular magnetic anisotropy modulation of $7000 fJ/V\cdot m$ \cite{YuPMA2015}. Vise versa, it has the high potential for voltage control of magnetoelastic coupling, towards the modulation of nonreciprocity.

Meanwhile, the MgO/CoFeB/Ta trilayer system provides another possibility for steering the nonreciprocity, DMI. In this trilayer system, within a particular thickness range of the CoFeB, thin film exhibits DMI at the CoFeB/Ta interface. In the magnonic community, DMI is one of the most well-established mechanisms inducing the nonreciprocity of spin waves. Also, for the magnetoelastic coupling, it has been theoretically demonstrated DMI induces nondegeneracy of the magnetoelastic band gap for the inversion of acoustic wave propagation direction \cite{VerbaNRMSDMI2018}.

In short, in our experiment, we are going to pass surface acoustic waves through Mg/CoFeB/Ta trilayer both with and without the presence of DMI, and studying the nonreciprocity for the propagation of the waves (both spin wave and acoustic wave) while modulating the perpendicular magnetic anisotropy (PMA) by applying voltage. In the first stage of the characterization, we are planning to perform an electrical measurement same as the previous report \cite{XuIEEMS2018}. Characterizing the angular dependence of acoustic ferromagnetic resonance by vector network analyzer through piezoelectric effect, and inverse spin Hall effect by utilizing Ta layer. However, for these methods, the spatial resolution is limited by the position of the electrodes, and the dynamic information cannot be resolved. So, for the second stage, optical characterization seems to be an indispensable tool to complete the understanding of the nonreciprocity in the magnon-phonon coupling. Notably, the Brillouin light scattering spectroscopy (BLS) seems the best candidate to meet these requirements. It has highly precise spatial resolution and sensitivity for wave vector.  

%\begin{figure}
%\centering
%\includegraphics[width=0.3\textwidth]{frog.jpg}
%\caption{\label{fig:frog}This frog was uploaded via the project menu.}
%\end{figure}